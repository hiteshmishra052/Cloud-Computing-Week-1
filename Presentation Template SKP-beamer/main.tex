\documentclass{SKP-beamer}

% --------------------------------------------------- %
%                  Presentation info	              %
% --------------------------------------------------- %
\title[Cloud Computing]{CLOUD COMPUTING}
\subtitle{Cloud Computing - Overview}
\author{PROF.SOUMYA K.GHOSH}
\institute[IIT,KGP]{
  DEPARTMENT OF COMPUTER SCIENCE AND ENGINEERING IIT KHARAGPUR
}
\date{\today}
\logo{
\includegraphics[scale=0.008]{images/logo.png}
}
\subject{Presentation subject} % metadata

% --------------------------------------------------- %
%                    Title + Schedule                 %
% --------------------------------------------------- %

\begin{document}

\begin{frame}
  \titlepage
\end{frame}

\begin{frame}{Introduction}
  The ACM Computing Curricula 2005 defined "computing" as
  
  "In a general way, we can define computing to mean any goal-oriented activity  requiring,  benefiting  from,  or  creating  computers.  Thus, computing includes designing and building hardware and software systems for a wide range of purposes; processing, structuring, and managing various kinds of information; doing scientific studies using computers; making computer systems behave intelligently; creating and  using  communications  and  entertainment  media;  finding  and gathering information relevant to any particular purpose, and so on. The list is virtually endless, and the possibilities are vast."
  
\end{frame}



\begin{frame}{Cloud Computing Course - Overview}
	\begin{itemize}
		\item  Introduction to Cloud Computing
		\begin{itemize}
			\item  Overview of Computing
			\item Cloud Computing (NIST Model)
			\item Properties, Characteristics \& Disadvantages
			\item Role of Open Standards
			
		\end{itemize}
		\item Cloud Computing Architecture
			\begin{itemize}
			\item  Cloud computing stack
            \item  Service Models (XaaS)
		        \begin{itemize}
		        	\item  Infrastructure as a Service(IaaS)
		        	\item  Platform as a Service(PaaS)
		        	\item Software as a Service(SaaS)
		        \end{itemize}
			\item  Deployment Models
	    	\end{itemize}
		\item Service Management in Cloud Computing
		        	\begin{itemize}
		        	\item  Service Level Agreements(SLAs)
		        	\item  Cloud Economics
		        	 \end{itemize}
		 \item Resource Management in Cloud 
		 Computing
		 
	\end{itemize}
\end{frame}


\begin{frame}{Cloud Computing Course (contd.)}
	\begin{itemize}
		\item  Data Management in Cloud Computing
		\begin{itemize}
			\item  Looking at Data, Scalability \& Cloud Services
			\item Database \& Data Stores in Cloud
			\item Large Scale Data Processing
		\end{itemize}
		\item Cloud Security
		\begin{itemize}
			\item Infrastructure Security
			\item Data security and Storage
			\item Identity and Access Management
			\item Access Control, Trust, Reputation, Risk
		\end{itemize}
		\item Case Study on Open Source and Commercial Clouds, Cloud Simulator
		\item Research trend in Cloud Computing, Fog Computing
		
		
	\end{itemize}	
\end{frame}


% --------------------------------------------------- %
%                      Presentation                   %
% --------------------------------------------------- %



\begin{frame}{Trends in Computing}
		\begin{itemize}
		\item Distributed Computing
		\item Grid Computing
		\item Cluster Computing
		\item Utility Computing
		\item \textbf{Cloud Computing}
		
	\end{itemize}
\end{frame}

\section{Distributed Computing}


\begin{frame}{Centralized vs. Distributed Computing}
\includegraphics[scale=1.2]{1.png}
\end{frame}


\begin{frame}{Distributed Computing/System?}
	\begin{itemize}
		\item Distributed Computing
		\begin{itemize}
			\item  Field of computing science that studies distributed system
			\item Use of distributed systems to solve computational problems.		
		\end{itemize}
		\item Distributed Computing
		\begin{itemize}
			\item Wikipedia
			
			\begin{itemize}
				\item  There are several autonomous computational entities, 
				each of which has its own local memory.
				\item The entities communicate with each other by message 
				passing.		
			\end{itemize}		
			
			\item Operating System Concept
				\begin{itemize}
				\item The processors communicate with one another through various 
				communication lines, such as high-speed buses or telephone 
				lines.
				\item Each processor has its own local memory.		
			\end{itemize}
		\end{itemize}
		
		
	\end{itemize}
	
	\includegraphics[scale=1.2]{2.png}
	
\end{frame}


\begin{frame}{Example Distributed Systems}
	\begin{itemize}
		\item Internet
		\item ATM(bank) machines
		\item Intranets/Workgroups
		\item Computing landscape will soon consist of 
		ubiquitous network-connected devices
		
		
	\end{itemize}
\end{frame}


\begin{frame}{Computers in a Distributed System}
	\begin{itemize}
		\item Workstations : Computers used by end-users to perform 
		computing
		\item Server Systems: Computers which provide resources and
		services
		\item Intranets/Workgroups
		\item Personal Assistance Devices: Handheld computers connected to the system via a wireless communication link.
		
	\end{itemize}
\end{frame}

\begin{frame}{Common properties of Distributed Computing}
	\begin{itemize}
		\item Fault tolerance
		\begin{itemize}
			\item  When one or some nodes fails, the whole system can still work fine except performance.
			\item Need to check the status of each node
		\end{itemize}
		\item Each node play partial role
		\begin{itemize}
			\item Each computer has only a limited, incomplete view of the system
			\item Each computer may know only one part of the input.
		\end{itemize}
		\item Resource sharing
		\begin{itemize}
			\item Each user can share the computing power and storage resource in the system with other 
			users
		\end{itemize}
		\item Load Sharing
			\begin{itemize}
			\item Dispatching several tasks to each nodes can help share loading to the whole system.
		\end{itemize}
		\item Easy to expand
		\begin{itemize}
			\item We expect to use few time when adding nodes. Hope to spend no time if possible.
		\end{itemize}
		\item Performance
		\begin{itemize}
			\item Parallel computing can be considered a subset of distributed computing.
		\end{itemize}
	\end{itemize}
	
\end{frame}


\begin{frame}{Why Distributed Computing?}
	\begin{itemize}
		\item Nature of application
		\item Performance
		
	
         	-- Computing Intensive
		\begin{itemize}
		\item The task could consume a lot of time on computing. For 
		example, Computation of Pi value using Monte Carlo simulation
	\end{itemize}
	       -- Data Intensive
	       \begin{itemize}
	       	\item The task that deals with a large amount or large size of files. For 
	       	example, Facebook, LHC(Large Hadron Collider) experimental data 
	       	processing.
	       \end{itemize}
	     \item Robustness \\
	       -- No SPOF (Single Point Of Failure)\\
	       -- Other nodes can execute the same task executed on failed 
	       node.
	       
    \end{itemize}
\end{frame}

\begin{frame}{Distributed applications}
	\begin{itemize}
		\item Applications that consist of a set of processes that are distributed across a 
		network of machines and work together as an ensemble to solve a 
		common problem
		\item In the past, mostly “client-server”
		\begin{itemize}
			\item Resource management centralized at the server
		\end{itemize}
		
		\item “Peer to Peer” computing represents a movement towards more “truly”
		distributed applications
		
		
		
	\end{itemize}
\end{frame}

\begin{frame}{Clients invoke individual servers}
	\includegraphics[scale=1.2]{3.png}
\end{frame}


\begin{frame}{A typical distributed application based on peer processes}
	\includegraphics[scale=1.4]{4.png}
\end{frame}



\section{\textbf{THANK YOU!!}}





\section{\textbf{Grid Computing}}

\begin{frame}{Grid Computing?}
	\begin{itemize}
		\item Pcwebopedia.com \\
		 – A form of networking. unlike conventional networks that focus on communication 
		among devices, grid computing harnesses unused processing cycles of all computers in 
		a network for solving problems too intensive for any stand-alone machine.
		
		\item IBM \\
		– Grid computing enables the virtualization of distributed computing and data resources 
		such as processing, network bandwidth and storage capacity to create a single system 
		image, granting users and applications seamless access to vast IT capabilities. Just as 
		an Internet user views a unified instance of content via the Web, a grid user essentially 
		sees a single, large virtual computer.
		
		
		\item Sun Microsystems \\
		
		– Grid Computing is a computing infrastructure that provides dependable, 
		consistent, pervasive and inexpensive access to computational capabilities
		
		
		
	\end{itemize}
\end{frame}


\begin{frame}{Grid Computing}
	\begin{enumerate}
		
		\item  Share more than information: Data, computing power, applications in 
		dynamic environment, multi-institutional, virtual organizations \\
		\item  Efficient use of resources at many institutes. People from many institutions
		working to solve a common problem (virtual organisation). \\
		\item  Join local communities. \\
		\item  Interactions with the underneath layers must be transparent and seamless 
		to the user.
		
	\end{enumerate}
\end{frame}


\begin{frame}{ Need Of Grid Computing?}
	\begin{itemize}
		
		\item   Today’s Science/Research is based on computations, data analysis, data 
		visualization \& collaborations
		\item Computer Simulations \& Modelling are more cost effectivethan
		experimental methods
		\item  JScientific and Engineering problems are becoming more complex \& users 
		need more accurate, precise solutions to their problems in shortest possible 
		time
		\item  Data Visualization is becoming very important
		\item Exploiting under utilized resources
		
	\end{itemize}
\end{frame}



\begin{frame}{Who uses Grid Computing ?}
	\includegraphics[scale=1.0]{5.png}
\end{frame}

\begin{frame}{ Types Of Grids}
	\begin{itemize}
		
		\item  \textbf{Computational Grid:} These grids provide secure access to huge pool of shared processing 
		power suitable for high throughput applications and computation intensive computing.
		\item \textbf{Data Grid:} Data grids provide an infrastructure to support data storage, data discovery, data 
		handling, data publication, and data manipulation of large volumes of data actually stored 
		in various heterogeneous databases and file systems.
		\item  \textbf{Collaboration Grid:} With the advent of Internet, there has been an increased demand for 
		better collaboration. Such advanced collaboration is possible using the grid. For instance, 
		persons from different companies in a virtual enterprise can work on different components of 
		a CAD project without even disclosing their proprietary technologies
		
	\end{itemize}
\end{frame}

\begin{frame}{}
	\begin{itemize}
		
		\item \textbf{Network Grid:} A Network Grid provides fault-tolerant and high-performance communication 
		services. Each grid node works as a data router between two communication points, 
		providing data-caching and other facilities to speed up the communications between such 
		points.
		
		\item  \textbf{Utility Grid:} This is the ultimate form of the Grid, in which not only data and computation 
		cycles are shared but software or just about any resource is shared. The main services 
		provided through utility grids are software and special equipment. For instance, the 
		applications can be run on one machine and all the users can send their data to be 
		processed to that machine and receive the result back
		
	\end{itemize}
\end{frame}

\begin{frame}{Grid Components}
	\includegraphics[scale=1.2]{6.png}
\end{frame}


\section{\textbf{Cluster Computing}}



\begin{frame}{What is Cluster Computing?}
	\begin{itemize}
		
		\item   A cluster is a type of parallelor distributed
		computer system, which consists of a
		collection of inter-connected stand-alone computers 
		working together as a single integrated
		computing resource .
		\item Key components of a cluster include multiple 
		standalone computers (PCs, Workstations, or SMPs), 
		operating systems, high-performance interconnects, 
		middleware, parallel programming environments, and
		applications.
		
	\end{itemize}
\end{frame}

\begin{frame}{ Cluster Computing}
	\begin{itemize}
		
		\item   Clusters are usually deployed to improve speed and/or reliability 
		over that provided by a single computer, while typically being 
		much more cost effective than single computer the of comparable 
		speed or reliability
		\item In a typical cluster \itemsep=3ex
		– Network: Faster, closer connection than a typical 
		network (LAN)
		– Low latency communication protocols
		– Loosely coupled than SMP
	\end{itemize}
\end{frame}


\begin{frame}{Types of Cluster}
	\begin{itemize}
		
		\item High Availability or Failover Clusters
		\item Load Balancing Cluster
		\item Parallel/Distributed Processing 
		Clusters
		
		
	\end{itemize}
\end{frame}


\begin{frame}{Cluster Components}
	\begin{itemize}
		
		\item Basic building blocks of clusters are broken down into 
		multiple categories :
		\item \textbf{Cluster Nodes}
		\item \textbf{Cluster Network}
		\item \textbf{Network Characterization}
		
			\includegraphics[scale=0.5]{7.png}
		
		
	\end{itemize}
\end{frame}


\begin{frame}{Key Operational Benefits of Clustering}
	\begin{itemize}
		
		\item   System availability: offer inherent high system availability due to the redundancy of hardware, operating systems, and applications.
		\item Hardware fault tolerance: redundancy for most system 
		components (eg. disk-RAID), including both hardware and 
		software.
		\item OS and application reliability: run multiple copies of the OS	and applications, and through this redundancy
		\item Scalability. adding servers to the cluster or by adding more clusters to the network as the need arises or CPU to SMP.
		
	\end{itemize}
\end{frame}

\section{\textbf{Utility Computing}}

\begin{frame}{“Utility” Computing?}
	\begin{itemize}
		
		\item Utility Computing is purely a concept which cloud computing practically implements.
		\item Utility computing is a service provisioning model in which a service provider makes 
		computing resources and infrastructure management available to the customer as 
		needed, and charges them for specific usage rather than a flat rate.
		\item This model has the advantage of a low or no initial cost to acquire computer resources; 
		instead, computational resources are essentially rented.
		\item The word utility is used to make an analogy to other services, such as electrical power, 
		that seek to meet fluctuating customer needs, and charge for the resources based on 
		usage rather than on a flat-rate basis. This approach, sometimes known as pay-per-use
		
		
	\end{itemize}
\end{frame}



\begin{frame}{Utility Computing Example}
	\begin{itemize}
		
		\item On-Demand Cyber
		\item Infrastructure
		
			\includegraphics[scale=1.5]{9.png}
		
		
		
	\end{itemize}
\end{frame}

\begin{frame}{Utility Solution – Your 
		Perspective Consumer Provider}
	
		
		\includegraphics[scale=1.5]{10.png}
		
		
		

\end{frame}



\begin{frame}{ Utility Computing Payment Models}
	\begin{itemize}
		
		\item   Same range of charging models as other utility providers: gas, electricity, telecommunications, water, 
		television broadcasting \\
		 - Flat rate \\
		 - Tiered \\
		 - Subscription \\
		 - Metered \\
		 - Pay as you go \\
		 - Standing charges \\
		\item Different pricing models for different customers based on factors such as scale, commitment and 
		payment frequency
		\item But the principle of utility computing remains
		\item The pricing model is simply an expression by the provider of the costs of provision of the resources and a 
		profit margin
		
		
		
		
		
	\end{itemize}
\end{frame}

%% RIYA'S CODE

\begin{frame}{ Risks in a UC World}
	\begin{itemize}
		
		\item  Data Backup
		\item Data Security
		\item Partner Competency
		\item Defining SLA
		\item Getting value from charge back
		
	\end{itemize}
\end{frame}


\section{\textbf{Cloud Computing}}

\begin{frame}{Cloud Computing}
	\begin{itemize}
		\item US National Institute of Standards and Technology defines Computing as:
		\item Cloud computing is a model for enabling ubiquitous, convenient, on-demand network access to a shared pool of 
		configurable computing resources (e.g networks, servers, storage, applications, and services) that can be 
		rapidly provisioned and released with minimal management effort or service provider interaction. ”
		
		\includegraphics[scale=1.5]{11.png}
		
	\end{itemize}
\end{frame}

\section{\textbf{Thank You!!}}

\section{\textbf{Cloud Computing}}



\begin{frame}{Cloud Computing}
	\begin{itemize}
		\item US \textbf{National Institute of Standards and Technology (NIST)} defines Computing as:
		\item Cloud computing is a model for enabling ubiquitous, convenient, on-demand network access to a shared pool of 
		configurable computing resources (e.g networks, servers, storage, applications, and services) that can be 
		rapidly provisioned and released with minimal management effort or service provider interaction. ”
		
		\includegraphics[scale=1.5]{11.png}
		
	\end{itemize}
\end{frame}


\begin{frame}{ Essential Characteristics}
	\begin{itemize}
		
		\item  \textbf{On-demand self-service}
		\begin{itemize}
			
			\item A consumer can unilaterally provision computing capabilities, such as server time and network storage, as 
			needed automatically without requiring human interaction with each service provider.
			
		\end{itemize}
		\item \textbf{Broad network access}
		\begin{itemize}
			
			\item Capabilities are available over the network and accessed through standard mechanisms that promote use by 
			heterogeneous thin or thick client platforms (e.g., mobile phones, tablets, laptops, and workstations).
		\end{itemize}
		\item \textbf{Resource pooling}
		\begin{itemize}
			
			\item The provider’s computing resources are pooled to serve multiple consumers using a multi-tenant model, 
			with different physical and virtual resources dynamically assigned and reassigned according to consumer 
			demand.
			
		\end{itemize}
	\end{itemize}
\end{frame}


\begin{frame}{ Cloud Characteristics}
	\begin{itemize}
		
		\item  \textbf{Measured Service}
		\begin{itemize}
			\item Cloud systems automatically control and optimize resource use by leveraging a metering 
			capability at some level of abstraction appropriate to the type of service (e.g., storage, 
			processing, bandwidth, and active user accounts). Resource usage can be
			\item monitored, controlled, and reported, providing transparency for both the provider and 
			consumer of the utilized service.
			
		\end{itemize}
		\item \textbf{Rapid elasticity}
		\begin{itemize}
			
			\item Capabilities can be elastically provisioned and released, in some cases automatically, 
			to scale rapidly outward and inward commensurate with demand. To the consumer, 
			the capabilities available for provisioning often appear to be unlimited and can be appropriated in any quantity at any time.
		\end{itemize}
	\end{itemize}
\end{frame}


\begin{frame}{ Common Characteristics}
	\begin{itemize}
		
		\item Massive Scale
		\item Resilient Computing Homogeneity
		\item Geographic Distribution
		\item Virtualization
		\item Service Orientation
		\item Low Cost Software
		\item Advanced Security
		
		
	\end{itemize}
\end{frame}


\begin{frame}{Cloud Services Models}
	\begin{itemize}
		
		\item  \textbf{Software as a Service (SaaS)}
		\begin{itemize}
			
			\item The capability provided to the consumer is to use the provider’s applications running on a cloud infrastructure. The applications 
			are accessible from various client devices through either a thin client interface, such as a web browser (e.g., web-based email), or 
			a program interface.
			
			\item The consumer does not manage or control the underlying cloud infrastructure including network, servers, operating systems, 
			storage, or even individual application capabilities, with the possible exception of limited user-specific application configuration 
			settings.
			
			\item e.g: Google Spread Sheet
			
		\end{itemize}
		\item \textbf{Cloud Infrastructure as a Service (IaaS)}
		\begin{itemize}
			
			\item The capability provided to provision processing, storage, networks, and other fundamental computing resources
			\item Consumer can deploy and run arbitrary software
			\item e.g: Amazon Web Services and Flexi scale.
		\end{itemize}
	\end{itemize}
\end{frame}


\begin{frame}{Cloud Services Models}
	\begin{itemize}
		
		\item  \textbf{Platform as a Service (PaaS)}
		\begin{itemize}
			
			\item The capability provided to the consumer is to deploy onto the cloud infrastructure consumer-created or 
			acquired applications created using programming languages, libraries, services, and tools supported by the 
			provider.
			
			\item The consumer does not manage or control the underlying cloud infrastructure including network, servers, 
			operating systems, or storage, but has control over the deployed applications and possibly configuration 
			settings for the application-hosting environment.
			
		\end{itemize}
	\end{itemize}
\end{frame}

\begin{frame}{Cloud Services Models}
	
	\includegraphics[scale=1.5]{12.png}
	
\end{frame}



\begin{frame}{Types of Cloud (Deployment Models)}
	\begin{itemize}
		
		\item  \textbf{Private Cloud} \\
		The cloud infrastructure is operated solely for an organization.
		e.g Window Server 'Hyper-V'. \\
		
		\item  \textbf{Community Cloud} \\
		The cloud infrastructure is shared by several organizations and supports a specific goal.
		
		\item \textbf{Public cloud} \\
		The cloud infrastructure is made available to the general public
		e.g Google Doc, Spreadsheet.
		\item \textbf{Hybrid Cloud} \\
		The cloud infrastructure is a composition of two or more clouds (private, community, or public)
		e.g Cloud Bursting for load balancing between clouds.
		
	\end{itemize}
\end{frame}

\begin{frame}{Cloud and Virtualization}
	\begin{itemize}
		
		\item  \textbf{Virtual Workspaces:} 
		\begin{itemize}
			\item An abstraction of an execution environment that can be made dynamically available to
			authorized clients by using well-defined protocols,
			\item Resource quota (e.g. CPU, memory share),
			\item Software configuration (e.g. OS)
			
		\end{itemize}
		\item  \textbf{Implement on Virtual Machines (VMs):} 
		\begin{itemize}
			\item  Abstraction of a physical host machine
			\item Hypervisor intercepts and emulates instructions from VMs, and allows management of VMs
			\item VMWare, Xen, KVM etc.
			
		\end{itemize}
		\item  \textbf{Provide infrastructure API:} 
		\begin{itemize}
			\item  Plug-ins to hardware/support 
			structures 
		\end{itemize}
		
	\end{itemize}
	
	\includegraphics[scale=1.5]{13.png}
	
\end{frame}


\begin{frame}{Cloud and Virtualization}
	\begin{itemize}
		
		\item  VM technology allows multiple virtual machines to run on a single
		physical machine.
		
		\includegraphics[scale=1.2]{14.png}
		
		\item Performance: Para-virtualization (e.g. Xen) is very close to raw physical
		performance!
	\end{itemize}
\end{frame}


\begin{frame}{Virtualization in General}
	\begin{itemize}
		
		\item  \textbf{Advantages of virtual machines:}
		\begin{itemize}
			\item  Run operating systems where the physical hardware is unavailable,
			\item  Easier to create new machines, backup machines, etc.,
			\item Software testing using “clean” installs of operating systems and software,
			\item  Emulate more machines than are physically available,
			\item  Timeshare lightly loaded systems on one host,
			\item  Debug problems (suspend and resume the problem machine),
			\item  Easy migration of virtual machines (shutdown needed or not).
			\item  Run legacy systems
			
			
		\end{itemize}
	\end{itemize}
\end{frame}


\begin{frame}{Cloud-Sourcing}
	\begin{itemize}
		
		\item  \textbf{Why is it becoming important ?}
		\begin{itemize}
			\item  Using high-scale/low-cost providers,
			\item  Any time/place access via web browser,
			\item  Rapid scalability; incremental cost and load sharing,
			\item  Can forget need to focus on local IT.
		\end{itemize}
		\item  \textbf{Concerns:}
		\begin{itemize}
			\item  Performance, reliability, and SLAs,
			\item  Control of data, and service parameters,
			\item  Application features and choices,
			–\item Interaction between Cloud providers,
			\item  No standard API – mix of SOAP and REST!
			\item  Privacy, security, compliance, trust…
			
		\end{itemize}
	\end{itemize}
\end{frame}



\begin{frame}{Cloud-Storage}
	\begin{itemize}
		
		\item  Several large Web companies are now exploiting the fact that they have data
		storage capacity that can be hired out to others.
		\begin{itemize}
			\item  Allows data stored remotely to be temporarily cached on 
			desktop computers, mobile phones or other 
			Internet-linked devices.
		\end{itemize}
		\item  Amazon’s Elastic Compute Cloud (EC2) and Simple Storage Solution (S3) are well
		known examples
	\end{itemize}
\end{frame}

\begin{frame}{Advantages of Cloud Computing}
	\begin{itemize}
		
		\item  \textbf{Lower computer costs:}
		\begin{itemize}
			\item  No need of a high-powered and high-priced computer to run cloud computing's 
			web-based applications.
			\item  Since applications run in the cloud, not on the desktop PC, your desktop PC does not 
			need the processing power or hard disk space demanded by traditional desktop 
			software.
			\item  When you are using web-based applications, your PC can be less expensive, with a
			smaller hard disk, less memory, more efficient processor...
			\item  In fact, your PC in this scenario does not even need a CD or DVD drive, as no 
			software programs have to be loaded and no document files need to be saved.
			
		\end{itemize}
		
	\end{itemize}
\end{frame}
\begin{frame}{Advantages of Cloud Computing}
	\begin{itemize}
		
		\item  \textbf{Improved performance:}
		\begin{itemize}
			\item  With few large programs hogging your computer's memory, you will see better 
			performance from your PC.
			\item  Computers in a cloud computing system boot and run faster because they have 
			fewer programs and processes loaded into memory.
			
		\end{itemize}
		\item  \textbf{Reduced software costs:}
		\begin{itemize}
			\item  Instead of purchasing expensive software applications, you can get most of what 
			you need for free. \\
			• most cloud computing applications today, such as the Google Docs suite.
			\item  better than paying for similar commercial software \\
			• which alone may be justification for switching to cloud applications.
			
			
		\end{itemize}
	\end{itemize}
\end{frame}

\begin{frame}{Advantages of Cloud Computing}
	\begin{itemize}
		
		\item  \textbf{Instant software updates}
		\begin{itemize}
			\item  Another advantage to cloud computing is that you are no longer faced with choosing
			between obsolete software and high upgrade costs.
			\item  When the application is web-based, updates happen automatically available the next time 
			you log into the cloud.
			\item  When you access a web-based application, you get the latest version without needing to pay 
			for or download an upgrade.
			
		\end{itemize}
		\item  \textbf{Improved document format compatibility.}
		\begin{itemize}
			\item  You do not have to worry about the documents you create on your machine being 
			compatible with other users' applications or OS.
			\item  There are less format incompatibilities when everyone is sharing documents and 
			applications in the cloud.
			
		\end{itemize}
	\end{itemize}
\end{frame}






\begin{frame}{Advantages of Cloud Computing}
	\begin{itemize}
		
		\item  \textbf{Unlimited storage capacity}
		\begin{itemize}
			\item  Cloud computing offers virtually limitless storage.
			\item  Your computer's current 1 Tera Bytes hard drive is small compared to the hundreds of Peta 
			Bytes available in the cloud.
			
		\end{itemize}
		\item  \textbf{Increased data reliability}
		\begin{itemize}
			\item  Unlike desktop computing, in which if a hard disk crashes and destroy all your 
			valuable data, a computer crashing in the cloud should not affect the storage of your 
			data. \\
			• if your personal computer crashes, all your data is still out there in the cloud, still accessible
			\item  In a world where few individual desktop PC users back up their data on a regular basis, 
			cloud computing is a data-safe computing platform. For e.g. Dropbox, Skydrive
			
		\end{itemize}
	\end{itemize}
\end{frame}

\begin{frame}{Advantages of Cloud Computing}
	\begin{itemize}
		
		\item  \textbf{Universal information access}
		\begin{itemize}
			\item  That is not a problem with cloud computing, because you do not take your
			documents with you.
			\item  Instead, they stay in the cloud, and you can access them whenever you have a 
			computer and an Internet connection
			\item  Documents are instantly available from wherever you are.
			
		\end{itemize}
		\item  \textbf{Latest version availability}
		\begin{itemize}
			\item  When you edit a document at home, that edited version is what you see when
			you access the document at work.
			\item  The cloud always hosts the latest version of your documents as long as you are 
			connected, you are not in danger of having an outdated version
			
		\end{itemize}
	\end{itemize}
\end{frame}


\begin{frame}{Advantages of Cloud Computing}
	\begin{itemize}
		
		\item  \textbf{Easier group collaboration}
		\begin{itemize}
			\item  Sharing documents leads directly to better collaboration.
			\item  Many users do this as it is an important advantages of cloud computing
			multiple users can collaborate easily on documents and projects
			
		\end{itemize}
		\item  \textbf{Device independence}
		\begin{itemize}
			\item  You are no longer tethered to a single computer or network.
			\item  Changes to computers, applications and documents follow you through the
			cloud.
			\item  Move to a portable device, and your applications and documents are still 
			available
			
		\end{itemize}
	\end{itemize}
\end{frame}

\begin{frame}{Disadvantages of Cloud Computing}
	\begin{itemize}
		
		\item  \textbf{Requires a constant internet connection}
		\begin{itemize}
			\item  Cloud computing is impossible if you cannot connect to the Internet.
			\item  Since you use the Internet to connect to both your applications and documents, if you do not 
			have an Internet connection you cannot access anything, even your own documents.
			\item  A dead Internet connection means no work and in areas where Internet connections are few or 
			inherently unreliable, this could be a deal-breaker
			
		\end{itemize}
		\item  \textbf{Does not work well with low-speed connections}
		\begin{itemize}
			\item  Similarly, a low-speed Internet connection, such as that found with dial-up services, makes 
			cloud computing painful at best and often impossible.
			\item  Web-based applications require a lot of bandwidth to download, as do large documents
			
		\end{itemize}
	\end{itemize}
\end{frame}



\begin{frame}{Disadvantages of Cloud Computing}
	\begin{itemize}
		
		\item  \textbf{Features might be limited}
		\begin{itemize}
			\item  This situation is bound to change, but today many web-based applications simply 
			are not as full-featured as their desktop-based applications. \\
			• For example, you can do a lot more with Microsoft PowerPoint than with Google 
			Presentation's web-based offering
			
		\end{itemize}
		\item  \textbf{ Can be slow}
		\begin{itemize}
			\item Even with a fast connection, web-based applications can sometimes be slower than 
			accessing a similar software program on your desktop PC.
			\item  Everything about the program, from the interface to the current document, has to
			be sent back and forth from your computer to the computers in the cloud.
			\item  If the cloud servers happen to be backed up at that moment, or if the Internet is 
			having a slow day, you would not get the instantaneous access you might expect 
			from desktop applications.
			
		\end{itemize}
	\end{itemize}
\end{frame}


\begin{frame}{Disadvantages of Cloud Computing}
	\begin{itemize}
		
		\item  \textbf{Stored data might not be secured}
		\begin{itemize}
			\item  With cloud computing, all your data is stored on the cloud. \\
			• The questions is How secure is the cloud?
			\item  Can unauthorized users gain access to your confidential data ?
			
		\end{itemize}
		\item  \textbf{Stored data can be lost!}
		\begin{itemize}
			\item  Theoretically, data stored in the cloud is safe, replicated across multiple machines.
			\item  But on the off chance that your data goes missing, you have no physical or local backup. \\
			• Put simply, relying on the cloud puts you at risk if the cloud lets you down.
			
			
		\end{itemize}
	\end{itemize}
\end{frame}

\begin{frame}{Disadvantages of Cloud Computing}
	\begin{itemize}
		
		\item  \textbf{HPC Systems}
		\begin{itemize}
			\item  Not clear that you can run compute-intensive HPC applications that use MPI/OpenMP!
			\item  Scheduling is important with this type of application \\
			• as you want all the VM to be co-located to minimize communication latency!
			
		\end{itemize}
		\item  \textbf{General Concerns}
		\begin{itemize}
			\item  Each cloud systems uses different protocols and different APIs \\
			• may not be possible to run applications between cloud based systems
			\item  Amazon has created its own DB system (not SQL 92), and workflow system (many 
			popular workflow systems out there)\\
			• so your normal applications will have to be adapted to execute on these platforms.
			
			
		\end{itemize}
	\end{itemize}
\end{frame}

%% RIYA'S CODE ENDS

% ------------------Hitesh Codes--------------------- %
\section{\textbf{Evolution of Cloud Computing
}}
\begin{frame}{ Reasons}
	\begin{itemize}
		\item The main reason for interest in cloud computing is due to the fact that public clouds can significantly reduce IT costs.
		
		\item From and end user perspective cloud computing gives the illusion of potentially infinite capacity with ability to scale rapidly and pay only for the consumed resource
		\item In contrast, provisioning for peak capacity is a necessity within private data centers, leading to a low average utilization of 5-20 percent.	
	\end{itemize}
\end{frame}

\begin{frame}{IaaS Economics}
	\includegraphics[scale=0.8]{a.png}
\end{frame}


\begin{frame}{Benefits for the end user while using public cloud}
	\begin{itemize}
		\item High Utilization
		
		\item High Scalability
		\item No separate hardware procurement
		\item No separate power cost
		\item No separate IT infrastructure administration/maintenance required
		\item Public clouds offer user friendly SLA by offering high availability (~99percent) and also provide compensation in case of SLA miss. 
		\item Users can rent the cloud to develop and test prototypes before making major
		investments in technology
		
	\end{itemize}
\end{frame}

\begin{frame}{Benefits for the end user while using public cloud}
	\begin{itemize}
		\item In order to enhance portability from one public cloud to another, several organizations such as Cloud Computing Interoperability Forum and Open Cloud Consortium are coming up with standards for portability.
		\item For e.g. Amazon EC2 and Eucalyptus share the same API interface.
		\item Software startups benefit tremendously by renting computing and storage infrastructure on the cloud instead of buying them as they are uncertain about their own future.	
	\end{itemize}
\end{frame}

\begin{frame}{Benefits for Small and Medium Businesses (<250 employees)}
	\includegraphics[scale=0.8]{b.png}
\end{frame}


\begin{frame}{Benefits of private cloud}
	\begin{itemize}
		\item Cost of 1 server with 12 cores and 12 GB RAM is far lower than the cost of 12 servers having 1 core and 1 GB RAM.
		\item Confidentiality of data is preserved
		\item Virtual machines are cheaper than actual machines
		\item Virtual machines are faster to provision than actual machines	
	\end{itemize}
\end{frame}

\begin{frame}{Economics of PaaS vs IaaS}
	\begin{itemize}
		\item Consider a web application that needs to be available 24X7, but
		where the transaction volume is unpredictable and can vary rapidly
		\item Using an IaaS cloud, a minimal number of servers would need to be
		provisioned at all times to ensure availability
		\item In contrast, merely deploying the application on PaaS cloud costs nothing. Depending upon the usage, costs are incurred.
		\item The PaaS cloud scales automatically to successfully handle increased requests to the web application.
		
	\end{itemize}
\end{frame}


\begin{frame}{PaaS benefits}
	\begin{itemize}
		\item No need for the user to handle scaling and load balancing of requests among virtual machines
		\item PaaS clouds also provide web based Integrated Development Environment for development and deployment of application on the PaaS cloud.
		\item Easier to migrate code from development environment to the actual production environment.
		\item Hence developers can directly write applications on the cloud and
		don’t have to buy separate licenses of IDE.	
	\end{itemize}
\end{frame}

\begin{frame}{SaaS benefits}
	\begin{itemize}
		\item Users subscribe to web services and web applications instead of buying and licensing software instances.
		\item For e.g. Google Docs can be used for free, instead of buying document reading softwares such as Microsoft Word.
		\item Enterprises can use web based SaaS Content Relationship Management	applications, instead of buying servers and installing CRM softwares and associated databases on them.
		
	\end{itemize}
\end{frame}

\begin{frame}{Benefits, as perceived by the IT industry}
	\includegraphics[scale=0.7]{c.png}
\end{frame}

\begin{frame}{Factors driving investment in cloud}
	\includegraphics[scale=0.6]{d.png}
\end{frame}

\begin{frame}{Factors driving investment in cloud}
	\includegraphics[scale=0.8]{e.png}
\end{frame}

\begin{frame}{Purpose of cloud computing in organizations}
	\begin{itemize}
		\item Providing an IT platform for business processes involving multiple organizations
		\item Backing up data
		\item Running CRM, ERP, or supply chain management applications
		\item Providing personal productivity and collaboration tools to employees
		\item Developing and testing software
		\item Storing and archiving large files (e.g., video or audio)
		\item Analyzing customer or operations data
		\item Running e-business or e-government web sites	
	\end{itemize}
\end{frame}

\begin{frame}{Purpose of cloud computing in organizations}
	\begin{itemize}
		\item Analyzing data for research and development
		\item Meeting spikes in demand on our web site or internal systems
		\item Processing and storing applications or other forms
		\item Running data-intensive batch applications (e.g., data conversion, risk modeling, graphics rendering)
		\item Sharing information with the government or regulators
		\item Providing consumer entertainment, information and communication (e.g., music,
		video, photos, social networks)		
	\end{itemize}
\end{frame}

\begin{frame}{Top cloud applications that are driving cloud adaptation}
	\begin{itemize}
		\item Mail and Messaging
		\item Archiving
		\item Backup
		\item Storage
		\item Security
		\item Virtual Servers
		\item CRM (Customer Relationship Management)
		\item Collaboration across enterprises
		\item Hosted PBX (Private Branch Exchange)
		\item Video Conferencing
		
	\end{itemize}
\end{frame}

\section{\textbf{THANK YOU!!}}

\section{\textbf{CLOUD COMPUTING ARCHITECTURE}}

\begin{frame}{Context: High Level Architectural Approach}
	\includegraphics[scale=0.9]{f.png}
\end{frame}

\begin{frame}{ Cluster Computing}
	\begin{itemize}
		\item Technical Architecture: \\
		– Structuring according to XaaS stack \\
		– Adopting cloud computing paradigms \\
		– Structuring cloud services and cloud components \\
		– Showing relationships and external endpoints \\
		– Middleware and communication \\
		– Management and security
		
	    \item Deployment Operation Architecture: \\
	    -- Geo-location check (Legal issues, export control) \\
	    -- Operation and Monitoring
		
	\end{itemize}
\end{frame}


\begin{frame}{Cloud Computing Architecture - XaaS}
	\includegraphics[scale=0.6]{g.png}
\end{frame}

\begin{frame}{XaaS Stack views: Customer view vs Provider view}
	\includegraphics[scale=0.8]{h.png}
\end{frame}

\begin{frame}{Microsoft Azure vs Amazon EC2}
	\includegraphics[scale=0.8]{i.png}
\end{frame}

\begin{frame}{Architecture for elasticity}
	\includegraphics[scale=0.7]{j.png}
\end{frame}

\begin{frame}{Service Models (XaaS)}
	\begin{itemize}
		\item Combination of Service-Oriented Infrastructure (SOI) and cloud computing realizes to XaaS.
		\item X as a Service (XaaS) is a generalization for cloud-related services
		\item XaaS stands for "anything as a service" or "everything as a service“
		\item XaaS refers to an increasing number of services that are delivered over the Internet rather than provided locally or on-site
		\item XaaS is the essence of cloud computing.
	\end{itemize}
\end{frame}

\begin{frame}{Service Models (XaaS)}
	\includegraphics[scale=0.6]{k.png}
\end{frame}

\begin{frame}{Service Models (XaaS)}
	\includegraphics[scale=0.6]{l.png}
\end{frame}

\begin{frame}{Service Models (XaaS)}
	\begin{itemize}
		\item Most common examples of XaaS are \\
		-- Software as a Service (SaaS) \\
		-- Platform as a Service (PaaS) \\
		-- Infrastructure as a Service (IaaS) \\
		
		\item Other examples of XaaS include \\
		-- Business Process as a Service (BPaaS) \\
		-- Storage as a service (another SaaS) \\
		-- Security as a service (SECaaS) \\
		-- Database as a service (DaaS) \\
		-- Monitoring/management as a service (MaaS) \\
		-- Communications, content and computing as a service (CaaS) \\
		-- Identity as a service (IDaaS) \\
		-- Backup as a service (BaaS) \\
		-- Desktop as a service (DaaS) 
		
		
	\end{itemize}
\end{frame}

\begin{frame}{Requirements of CSP (Cloud Service Provider)}
	\begin{itemize}
		\item Increase Productivity
		\item Increase end user satisfaction
		\item Increase innovation
		\item Increase agility

	\end{itemize}
\end{frame} 

\begin{frame}{Service Models (XaaS)}
	\begin{itemize}
		\item Broad network access (cloud) + resource pooling (cloud) + business-driven infrastructure on-demand (SOI) + service- orientation (SOI) = XaaS
		\item Xaas fulfils all the 4 demands!
		\includegraphics[scale=0.7]{m.png}
	\end{itemize}
\end{frame}

\begin{frame}{Classical Service Model}
	\begin{itemize}
		\item All the Layers(H/W, Operating System, Development Tools, Applications) Managed by the Users
    	\item Initial IT budget and resources
    	\item Users bears the costs of the hardware, maintenance and technology
    	\item Each system is designed and funded for a specific business activity: custom build-to-order
    	\item Systems are deployed as a vertical stack of “layers” which are tightly coupled, so no single part can be easily replaced or changed
    	\item Prevalent of manual operations for provisioning
    	\item Result: Legacy IT
    	
    	
		\includegraphics[scale=0.7]{n.png}
	\end{itemize}
\end{frame}


\begin{frame}{Key impact of cloud computing for IT function:
		From Legacy IT to Evergreen IT}
	\includegraphics[scale=0.6]{o.png}
\end{frame}

\begin{frame}{Classic Model	vs.	XaaS}
	\includegraphics[scale=0.7]{p.png}
\end{frame}


\section{\textbf{THANK YOU!!}}

\section{\textbf{Cloud Computing Architecture}}

\begin{frame}{Client Server Architecture}
	\includegraphics[scale=0.8]{q.png}
\end{frame}

\begin{frame}{Client server architecture}
	\begin{itemize}
		\item Consists of one or more load balanced servers servicing requests sent by the clients
		\item Clients and servers exchange message in request-response fashion
		\item Client is often a thin client or a machine with low computational capabilities
		\item Server could be a load balanced cluster or a stand alone machine.
	\end{itemize}
\end{frame} 

\begin{frame}{Client Server Architecture}
	\includegraphics[scale=1.0]{r.png}
\end{frame}

\begin{frame}{Client Server model vs. Cloud model}
	\includegraphics[scale=0.7]{s.png}
\end{frame}

\begin{frame}{Cloud Services}
	\includegraphics[scale=0.7]{t.png}
\end{frame}

\begin{frame}{Cloud Services Models}
	\includegraphics[scale=0.7]{u.png}
\end{frame}

\begin{frame}{Simplified description of cloud service models}
	\begin{itemize}
		\item SaaS applications are designed for end users and are delivered over the web
		\item PaaS is the set of tools and services designed to make coding and deploying applications quickly and efficiently
		\item IaaS is the hardware and software that powers it all – servers, storage, network, operating systems
	\end{itemize}
\end{frame} 

\begin{frame}{Transportation Analogy}
	\begin{itemize}
		\item By itself, infrastructure isn’t useful – it just sits there waiting for someone to make it productive in solving a particular problem. Imagine the Interstate transportation system in
		the U.S. Even with all these roads built, they wouldn’t be
		useful without cars and trucks to transport people and goods. In this analogy, the roads are the infrastructure and the cars and trucks are the platform that sits on top of the infrastructure and transports the people and goods. These goods and people might be considered the software and information in the technical realm
	\end{itemize}
\end{frame}

\begin{frame}{Software as a Service}
	\begin{itemize}
		\item SaaS is defined as software that is deployed over the internet. With SaaS, a provider licenses an application to customers either as a service on demand, through a subscription, in a “pay-as-you-go” model, or (increasingly) at no charge when there is opportunity to generate revenue from streams other than the user, such as from advertisement or user list sales.
	\end{itemize}
\end{frame} 

\begin{frame}{SaaS characteristics}
	\begin{itemize}
		\item Web access to commercial software
		\item Software is managed from central location
		\item Software is delivered in a ‘one to many’ model
		\item Users not required to handle software upgrades and patches
		\item Application Programming Interfaces (API) allow for integration between different pieces of software
	\end{itemize}
\end{frame} 

\begin{frame}{Applications where SaaS is used}
	\begin{itemize}
		\item Applications where there is significant interplay between organization and outside world. E.g. email newsletter campaign software
		\item Applications that have need for web or mobile access. E.g. mobile sales management software
		\item Software that is only to be used for a short term need.
		\item Software where demand spikes significantly. E.g. Tax/Billing Softwares
		\item E.g. of SaaS: Sales Force Customer Relationship Management (CRM) software
	\end{itemize}
\end{frame} 


\begin{frame}{Applications where SaaS may not be the best option}
	\begin{itemize}
		\item Applications where extremely fast processing of real time data is needed
		\item Applications where legislation or other regulation does not permit data being hosted externally
		\item Applications where an existing on-premise solution fulfills all of the organization’s needs
	\end{itemize}
\end{frame} 

\begin{frame}{Platform as a Service}
	\begin{itemize}
		\item Platform as a Service (PaaS) brings the benefits that SaaS bought for applications, but over to the software development world. PaaS can be defined as a computing platform that allows the creation of web applications quickly and easily and without the complexity of buying and maintaining the software and infrastructure underneath it.
		\item PaaS is analogous to SaaS except that, rather than being software delivered over the web, it is a platform for the creation of software, delivered over the web
	\end{itemize}
\end{frame}

 \begin{frame}{Characteristics of PaaS}
 	\begin{itemize}
 		\item Services to develop, test, deploy, host and maintain applications in the same integrated development environment. All the varying services needed to fulfill the application development process.
 		\item Web based user interface creation tools help to create, modify, test and deploy
 		different UI scenarios.
 		\item Multi-tenant architecture where multiple concurrent users utilize the same development application.
 		\item Built in scalability of deployed software including load balancing and failover.
 		\item Integration with web services and databases via common standards.
 		\item Support for development team collaboration – some PaaS solutions include project planning and communication tools.
 		\item Tools to handle billing and subscription management		
 	\end{itemize}
 \end{frame} 
 
 \begin{frame}{Scenarios where PaaS is used}
 	\begin{itemize}
 		\item PaaS is especially useful in any situation where multiple developers will be working on a development project or where other external parties need to interact with the development process
 		\item PaaS is useful where developers wish to automate testing and
 		deployment services.
 		\item The popularity of agile software development, a group of software development methodologies based on iterative and incremental development, will also increase the uptake of PaaS as it eases the difficulties around rapid development and iteration of software.	
 	\end{itemize}
 \end{frame} 
 
 \begin{frame}{Scenarios where PaaS is not ideal}
 	\begin{itemize}
 		\item Where the application needs to be highly portable in terms of where it is hosted.
 		\item Where proprietary languages or approaches would impact on the development process
 		\item Where a proprietary language would hinder later moves to another	provider – concerns are raised about vendor lock in
 		\item Where application performance requires customization of the underlying hardware and software 		
 	\end{itemize}
 \end{frame} 
 
 \begin{frame}{Infrastructure as a Service}
 	\begin{itemize}
 		\item Infrastructure as a Service (IaaS) is a way of delivering Cloud Computing infrastructure – servers, storage, network and operating systems
 		– as an on-demand service.
 		\item Rather than purchasing servers, software, datacenter space or network equipment, clients instead buy those resources as a fully outsourced service on demand. 		 		
 	\end{itemize}
 \end{frame} 
 
 
  \begin{frame}{Characteristics of IaaS}
 	\begin{itemize}
 		\item Resources are distributed as a service
 		\item Allows for dynamic scaling
 		\item Has a variable cost, utility pricing model
 		\item Generally includes multiple users on a single piece of hardware	 		 		
 	\end{itemize}
 \end{frame} 
 
 \begin{frame}{Scenarios where IaaS makes sense}
 	\begin{itemize}
 		\item Where demand is very volatile – any time there are significant spikes
 		and troughs in terms of demand on the infrastructure
 		\item For new organizations without the capital to invest in hardware
 		\item Where the organization is growing rapidly and scaling hardware would be problematic
 		\item Where there is pressure on the organization to limit capital expenditure and to move to operating expenditure
 		\item For specific line of business, trial or temporary infrastructural needs		 		 		
 	\end{itemize}
 \end{frame}

 
 \begin{frame}{Scenarios where IaaS may not be the best option}
 	\begin{itemize}
 		\item Where regulatory compliance makes the offshoring or outsourcing of data storage and processing difficult
 		\item Where the highest levels of performance are required, and on-premise or dedicated hosted infrastructure has the capacity to meet the organization’s needs		 		 		
 	\end{itemize}
 \end{frame}
 
 \begin{frame}{SaaS Providers}
 	\includegraphics[scale=0.7]{v.png}
 \end{frame}
 
 \begin{frame}{Feature comparison of PaaS providers}
 	\includegraphics[scale=0.7]{w.png}
 \end{frame}
 
 \begin{frame}{Feature comparison of IaaS providers}
 	\includegraphics[scale=0.7]{x.png}
 \end{frame}
 
  \begin{frame}{XaaS}
 	\includegraphics[scale=0.7]{y.png}
 \end{frame}
 
 \begin{frame}{Role of Networking in cloud computing}
 	\begin{itemize}
 		\item In cloud computing, network resources can be provisioned dynamically.
 		\item Some of the networking concepts that form the core of cloud computing are Virtual Local Area \item \item Networks, Virtual Private Networks and the different protocol layers.
 		\item Examples of tools that help in setting up different network topologies and facilitate various network configurations are OpenSSH, OpenVPN etc. 			 		 		
 	\end{itemize}
 \end{frame}
 
  \begin{frame}{Networking in different cloud models}
 	\includegraphics[scale=0.7]{z.png}
 \end{frame}
 
 \begin{frame}{Network Function Virtualization}
 	\begin{itemize}
 		\item Definition: “Network Functions Virtualisation aims to transform the way that network operators architect networks by evolving standard IT virtualisation technology to consolidate many network equipment types onto industry standard high volume servers, switches and storage, which could be located in Datacentres, Network Nodes and in the end user premises, as illustrated in Figure 1. It involves the implementation of network functions in software that can run on a range of industry standard server hardware, and that can be moved to, or instantiated in, various locations in the network as required, without the need for installation of new equipment.”			 		 		
 	\end{itemize}
 \end{frame}
 
  \begin{frame}{Network Function Virtualization}
 	\includegraphics[scale=0.9]{za.png}
 \end{frame}
 
 \section{\textbf{THANK YOU!!}}
% -------------Hitesh Codes ends here--------------------%


\end{document}
